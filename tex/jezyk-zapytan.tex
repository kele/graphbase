\documentclass[thesis.tex]{subfiles}

\begin{document}

\chapter{Język zapytań}

\section{Abstrakcyjne drzewo rozbioru}

\subsection{Zbiór typów}
Zbiór typów $T$ definiujemy następująco:
\begin{itemize}
    \item $Bool, Integer, Float, Void \in T$
    \item jeśli $\alpha \in T$ to
        $Set\langle \alpha \rangle$, $List\langle \alpha \rangle,
        Gen\langle \alpha \rangle \in T$
    \item jeśli $\alpha \in T$ to
        $Node\langle \alpha \rangle \in T$
    \item jeśli $\alpha \in T$ to
        $Edge\langle \alpha \rangle \in T$
    \item jeśli $\alpha \in T$ to
        $Arc\langle \alpha \rangle \in T$
    \item jeśli $Node\langle \alpha \rangle \in T$ oraz $Edge\langle \beta \rangle \in T$ to
        $Graph\langle \alpha, \beta \rangle \in T$
    \item jeśli $Node\langle \alpha \rangle \in T$ oraz $Arc\langle \beta \rangle \in T$ to
        $Digraph\langle \alpha, \beta \rangle \in T$
\end{itemize}


\subsection{Zbiór wyrażeń i reguły typowania}
Niech $S$ będzie zbiorem wyrażeń. Jeśli $e \in S$, $\alpha \in T$ to $\Gamma
\vdash e : \alpha$ oznacza ,,wyrażenie $e$ ma typ $\alpha$ w kontekście
$\Gamma$'', gdzie $\Gamma \subset\ Ident \times T$. $Ident$ jest zbiorem identyfikatorów.


Zbiór wyrażeń $S$ oraz reguły typowania definiujemy następująco:
\begin{itemize}
    \item liczby naturalne należą do $S$ i mają typ $Integer$, tzn. dla każdej
        liczby naturalnej $n$ i dowolnego kontekstu $\Gamma$ mamy $\Gamma
        \vdash n : Integer$
    \item $true, false \in S$ oraz $\Gamma \vdash true : Bool$ i $\Gamma \vdash false : Bool$
    \item jeśli $n \in \mathbb{N}$, $e_1, e_2, ..., e_n \in S$ oraz
        $\forall_{1 \leq i \leq n} \, \Gamma \vdash e_i : \alpha$ to
        ${e_1, e_2, ..., e_n} \in S$ oraz
        $\Gamma \vdash {e_1, e_2, ..., e_n} : Set\langle \alpha \rangle$
    \item jeśli $e_1, e_2 \in S$,
        $\alpha \in T$,
        $x \in Ident$,
        $\Gamma \vdash e_1 : Gen\langle \alpha \rangle$,
        $\Gamma, (x, \alpha) \vdash e_2 : Bool$ to
        $Forall(x, e_1, e_2) \in S$ oraz $\Gamma \vdash Forall(x, e_1, e_2) : Bool$
        Pod warunkiem, że $x$ nie występuje w żadnej parze należącej do $\Gamma$.
    \item jeśli $e_1, e_2 \in S$,
        $\alpha \in T$,
        $x \in Ident$,
        $\Gamma \vdash e_1 : Gen\langle \alpha \rangle$,
        $\Gamma, (x, \alpha) \vdash e_2 : Bool$ to
        $Exists(x, e_1, e_2) \in S$ oraz $\Gamma \vdash Exists(x, e_1, e_2) : Bool$.
        Pod warunkiem, że $x$ nie występuje w żadnej parze należącej do $\Gamma$.
\end{itemize}


Zbiór $S$ oraz reguły typowania zostają wzbogacone o dodatkowe funkcje takie
jak: $VerticesNum$ oznaczająca liczbę wierzchołków w grafie, $AddEdge$ dodająca
wierzchołek do grafu, itd. Ich arność jak i reguły typowania dla nich są
zdefiniowane naturalnie, stąd zostały pominięte w formalnej definicji $S$ i regułach typowania.

Definiując $S$ wraz ze zbiorem reguł typowania sprawiamy, że wszystkie
syntaktycznie poprawne wyrażenia są także wyrażeniami, które poprawnie się
typują.


\subsection{Koercje}
Dozwolone (niejawne) koercje między typami:
\begin{itemize}
    \item jeśli $\Gamma \vdash e : Set\langle \alpha \rangle$ to
        $\Gamma \vdash e : Gen\langle \alpha \rangle$
    \item jeśli $\Gamma \vdash e : List\langle \alpha \rangle$ to
        $\Gamma \vdash e : Gen\langle \alpha \rangle$
    \item jeśli $\Gamma \vdash e : Gen\langle \alpha \rangle$ to
        $\Gamma \vdash e : Set\langle \alpha \rangle$ oraz
        $\Gamma \vdash e : List\langle \alpha \rangle$
    \item jeśli $\Gamma \vdash e : Graph\langle \alpha, \beta \rangle$ to
        $\Gamma \vdash e : Graph\langle Void, Void \rangle$
    \item jeśli $\Gamma \vdash e : Digraph\langle \alpha, \beta \rangle$ to
        $\Gamma \vdash e : Digraph\langle Void, Void \rangle$
\end{itemize}

\end{document}
